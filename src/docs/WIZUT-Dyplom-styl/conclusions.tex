%%%%%%%%%%%%%%%%%%%%%%%%%%%%%%%%%%%%%%%%%
% Wnioski do pracy dyplomowej
% Szablon pracy dyplomowej
% Wydział Informatyki 
% Zachodniopomorski Uniwersytet Technologiczny w Szczecinie
% autor Joanna Kołodziejczyk (jkolodziejczyk@zut.edu.pl)
% Bardzo wczesnym pierwowzorem szablonu był
% The Legrand Orange Book
% Version 2.1 (26/09/2018)
%
% Modifications to LOB assigned by %JK
%%%%%%%%%%%%%%%%%%%%%%%%%%%%%%%%%%%%%%%%%


\chapter*{Podsumowanie}
Celem niniejszej pracy było opracowanie systemu rozpoznawania tablic rejestracyjnych na obrazach z kamery samochodowej z użyciem wybranych algorytmów z zakresu przetwarzania obrazów i uczenia maszynowego.
W celu wykonania wspomnianego systemu, zebrano zbiór uczący składający się z 10985 zdjęć zawierających 10301 tablic rejestracyjnych.
Opracowano system, w skład którego wchodzą moduł detekcji tablic oraz moduł rozpoznawania, znajdujących się na nich, znaków.
Do klasyfikacji wykorzystano algorytm RealBoost ze słabymi klasyfikatorami realizowanymi poprzez koszykowanie wartości funkcji logit.
Klasyfikator oparty został o cechy Haara.
Do rozpoznawania znaków wykorzystano bibliotekę Tesseract opartą o rekurencyjne sieci neuronowe LSTM\@.
Program zaimplementowano w środowisku Python.
Testy przeprowadzono na surowych nagraniach z kilku popularnych kamer samochodowych ze średniej półki.

W stosunku do istniejących rozwiązań, opracowany system nie potrzebuje specjalnej aparatury do działania.
W ramach testów wykazano, że nagrania pochodzące z wideorejestratorów kosztujących maksymalnie kilkaset złotych, są wystarczające.
W przeprowadzonych eksperymentach udowodniono poprawność działania opisywanego rozwiązania.
Przedstawiono słabości systemu, wynikające z niedoskonałości użytych algorytmów.
Zrealizowany system cechuje się ogólną dokładnością równą 99.77\% przy czym czułością 82\%.
Przetwarzanie jednej klatki obrazu z kamery trwa w przybliżeniu 1 sekundę.
Stworzone oprogramowanie zostało przygotowane w ten sposób, że nie wymaga wielu modyfikacji, aby zwiększyć szybkość lub jakość detekcji.
Można to zrealizować poprzez zmianę parametru liczby używanych cech do klasyfikacji.

Komercyjne rozwiązania zapewniają wyższą dokładność oraz są bardziej wydajne.
Należy jednak mieć na uwadze fakt, że są one realizowane przez firmy mające odpowiednie zasoby.
Poprawę systemu można by uzyskać przede wszystkim poprzez ulepszenie algorytmów, zarówno lokalizacji, jak i rozpoznawania znaków.
Dodatkowo, stosując kamerę o wyższej rozdzielczości, możliwe jest poprawienie operacji rozpoznawania znaków.
Do osiągnięcia wyższej wydajności systemu, możliwe jest użycie komputera o większej mocy obliczeniowej.
Poniższa praca udowadnia słuszność stosowania cech Haara w zadaniu lokalizacji tablic rejestracyjnych.
Również algorytm boostingu zapewnił znaczne przyspieszenie klasyfikacji obiektów przy jednoczesnym zachowaniu odpowiedniego poziomu dokładności.

Opracowany system mógłby zostać wykorzystany np.\ do przetwarzania nagrań z wideorejestratorów znajdujących się w radiowozach.
Rozpoznane tablice rejestracyjne mogłyby być sprawdzane w zewnętrznym systemie.
Przykładowo, możliwe byłoby sprawdzenie czy pojazd posiada aktualny przegląd techniczny.
Funkcjonariusze byliby natychmiastowo powiadamiani o wykrytej nieprawidłowości.
W konsekwencji doszłoby do zatrzymania potencjalnie niebezpiecznego pojazdu, co znacznie ograniczyłoby stwarzane niebezpieczeństwo na drogach.

%Podsumowanie pracy powinno na maksymalnie dwóch stronach przedstawić główne wyniki pracy dyplomowej. Struktura zakończenia to:
%\begin{enumerate}
%\item Przypomnienie celu i hipotez
%\item Co w pracy wykonano by cel osiągnąć (analiza, projekt, oprogramowanie, badania eksperymentalne)
%\item Omówienie głównych wyników pracy
%\item Jak wyniki wzbogacają dziedzinę
%\item Zamknięcie np. poprzez wskazanie dalszych kierunków badań.
%\end{enumerate}
