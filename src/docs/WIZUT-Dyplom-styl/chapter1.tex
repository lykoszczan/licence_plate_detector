%%%%%%%%%%%%%%%%%%%%%%%%%%%%%%%%%%%%%%%%%
% Szablon pracy dyplomowej
% Wydział Informatyki 
% Zachodniopomorski Uniwersytet Technologiczny w Szczecinie
% autor Joanna Kołodziejczyk (jkolodziejczyk@zut.edu.pl)
% Bardzo wczesnym pierwowzorem szablonu był
% The Legrand Orange Book
% Version 2.1 (26/09/2018)
%
% Modifications to LOB assigned by %JK
%%%%%%%%%%%%%%%%%%%%%%%%%%%%%%%%%%%%%%%%%

%----------------------------------------------------------------------------------------
%	CHAPTER 1
% 	author: Joanna Kolodziejczyk (jkolodziejczyk@zut.edu.pl)
%----------------------------------------------------------------------------------------

\chapter{Wprowadzenie teoretyczne}
\chaptermark{Wprowadzenie teoretyczne}

Na przestrzeni ostatnich lat stosowanie Systemów Automatycznego Rozpoznawania Tablic Rejestracyjnych (ARTR)(ang. \textit{Automatic Licence Plate Recognition} - ALPR) stało się znacznie bardziej powszechne.
W większości dużych miast istnieją parkingi, gdzie po umieszczeniu opłaty za postój, przy zbliżeniu się do wyjazdu, szlaban otwiera się automatycznie po rozpoznaniu numeru rejestracyjnego pojazdu, w którym się poruszamy.
W obecnych czasach wszystkie nowoczesne systemy do zarządzania i sterowania ruchem drogowym oparte są o technologie ARTR.
Instytucje takie jak służby drogowe, dzięki rejestrowanym i przetwarzanym w czasie rzeczywistym ogromnym ilościom danych, są w stanie odpowiednio szybko reagować na wydarzenia na drogach takie jak kolizje, korki lub innego rodzaju utrudnienia.
Innym z możliwych przykładów zastosowania wspomnianych systemów są odcinkowe pomiary prędkości, opłaty za przejazd płatnymi drogami lub wykrywanie kierowców łamiących przepisy.
Dzięki nieustannemu rozwojowi technologii i co raz wydajniejszym komputerom, systemy stają się tańszą i łatwiej dostępną alternatywą dla systemów opartych na RFID (ang. \textit{Radio-frequency identification}), które to wymagają specjalnej etykiety do prawidłowego działania.

Rozpoznawanie tablic rejestracyjnych jest techniką polegająca na wykryciu i odczytaniu znaków z tablicy rejestracyjnych na podstawie zarejestrowanego obrazu.
Do tego celu wykorzystywany jest aparat o wysokiej rozdzielczości oraz odpowiedni program komputerowy.
Oprogramowanie otrzymuje na wejściu cyfrową reprezentację obrazu.
Dla zdjęć kolorowych każdy piksel opisany jest wartościami z palety barw RGB reprezentującymi jego barwę oraz współrzędnymi umiejscowienia w obrazie.
Dla zdjęć monochromatycznych barwy opisywane są najczęściej za pomocą wartości luminacji obrazu.

W procesie automatycznego rozpoznawania tablic rejestracyjnych pozyskany obraz jest odpowiednio przetwarzany.
Przed przejściem do rozpoznawania, obraz często jest konwertowany do skali szarości i filtrowany za pomocą filtrów np.
(Gaussa lub średnio-przepustowego) w celu redukcji szumu.
W procesie tym można wyróżnić trzy etapy \cite{1688109}:
\begin{itemize}
    \item \textbf{detekcję} - określenie położenia tablicy rejestracyjnej w analizowanym obrazie
    \item \textbf{segmentację} - wyodrębnienie pojedynczych znaków na fragmencie obrazu ze zlokalizowaną tablicą
    \item \textbf{identyfikację} - rozpoznanie każdego ze znaków i przedstawienie ich w formie tekstowej, którą można później wykorzystać do dalszych działań w zależności od przeznaczenia systemu
\end{itemize}
\FloatBarrier
Na rysunku \ref{fig:schemat_lpr} przedstawiono graficzną reprezentację powyższego procesu.
\begin{figure}[!ht]
    \centering
    \includegraphics[scale=0.6]{Pictures/schemat_lpr.png}
    \caption{Etapy procesu automatycznego rozpoznawania tablic rejestracyjnych (źródło: opracowanie własne)}
    \label{fig:schemat_lpr}
\end{figure}
\FloatBarrier
Kolejne etapy korzystają z wyników uzyskanych w poprzednich krokach, co oznacza, że błąd powstały we wcześniejszej fazie, będzie rzutował na jakość działania całego systemu.
W wielu systemach zanim dojdzie do rozpoznawania tablicy rejestracyjnej, obraz jest wpierw odpowiednio przetwarzany.
Jednymi z najpowszechniej stosowanych czynności są skalowanie obrazu, modyfikacje jasności oraz redukcja zakłóceń.
W zależności od wymagań stawianych przed danym mechanizmem i środowiskiem jego działania, czynności te mogą znacznie się od siebie różnić.
Najbardziej podstawowe systemy wymagają, aby pojazd znajdował się nieruchomo w określonym miejscu.
Tego typu rozwiązania najczęściej stosowane są na parkingach, gdzie szlaban otwiera się po odczycie numerów rejestracyjnych pojazdu i potwierdzeniu opłaty za postój w zewnętrznej bazie danych.
Takie systemy pracują z reguły w środowisku o niskim poziomie zakłóceń wynikających z warunków atmosferycznych i oświetlenia.
Obecnie na rynku znajduje się wiele komercyjnych rozwiązań, które oferują wysoką dokładność (powyżej 95\%) dla tego rodzaju detekcji.
Taki rodzaj systemów ARTR nazywany systemami statycznymi.
Dużo większą złożonością charakteryzują się systemy dynamiczne, w których znacznie większą rolę odgrywają zakłócenia wynikające ze zmiennych warunków oświetlenia.
W obecnych czasach stworzenie dynamicznego systemu ARTR o wysokiej dokładności wciąż stanowi wyzwanie i jest tematem wielu prac naukowych.
Celem niniejszej pracy jest analizowanie obrazów pochodzących z kamery samochodowej, co zdecydowanie sprawia, że jest to system dynamiczny.
Poniżej przedstawiono najczęściej stosowane metody widzenia komputerowego w systemach ARTR.


\section{Przegląd istniejących metod detekcji tablic rejestracyjnych}
\index{Przegląd istniejących metod detekcji tablic rejestracyjnych}

Zgodnie ze słowikiem języka polskiego, definicja tablicy rejestracyjnej brzmi następująco:
\begin{definition}[Tablica rejestracyjna]
    Płytka zawierająca numery identyfikacyjne pojazdu, umieszczana z przodu i z tyłu pojazdu.
\end{definition}
Dla programu komputerowego powyższe zdanie nie jest zrozumiałe.
W zadaniu detekcji tablicy rejestracyjnej, wymagane jest, aby maszyna ``zrozumiała'' jakich obiektów należy szukać.
W kontekście detekcji, za definicję można uznać ``prostokątny obszar, z dużym zagęszczeniem horyzontalnych i wertykalnych krawędzi''\cite{824138}.
W oparciu o powyższe cechy zaprezentowano wiele algorytmów do rozwiązania zadania wykrywania tablic rejestracyjnych.
Część z nich wywodzi się z tradycyjnych metod widzenia komputerowego i metod głębokiego uczenia.
Każda z metod ma swoje zalety, ale również często ograniczenia.
W związku z tym, trudno jednoznacznie stwierdzić, która z metod jest najbardziej efektywna.

Detekcja numerów rejestracyjnych jest wyzywającym zadaniem ze względu na poniższe czynniki:
\begin{itemize}
    \item tablica rejestracyjna zajmuje niewielki obszar na zdjęciu
    \item istnienie ogromnej ilości formatów tablic rejestracyjnych (w zależności od kraju rejestracji lub rodzaju pojazdu)
    \item słabe oświetlenie, rozmazany obraz, refleksy świetlne
    \item ruch pojazdu, zabrudzone tablice
\end{itemize}
Tradycyjne metody widzenia komputerowego oparte są na cechach takich jak kształt, kolor, symetria, tekstury itp.\cite{9310202}.
W celu uzyskania lepszych wyników, spotyka się rozwiązania, w których łączy się wiele technik.
Poniżej wyróżniono najczęściej stosowane metody w detekcji tablic rejestracyjnych.

\subsection{Metody oparte na krawędziach (ang. \textit{Edge based)}}
Biorąc pod uwagę, że tablica rejestracyjna jest prostokątem o znanych proporcjach, większość badań bazuje na podejściu opartym o wykrywanie krawędzi.
W większości przypadków kolor tablicy rejestracyjnej jest różny od koloru pojazdu.
Dzięki temu, granice tablicy zostają uznane za krawędzie.
Wiele metod wykorzystuje filtr Sobela.
Jego działania polega dyskretnym różniczkowaniu i aproksymacji pochodnych kierunkowych intensywności obrazu.
Filtr ten składa się z dwóch macierzy o wymiarach 3x3 (\ref{fig:sobel_filter}) służących do detekcji krawędzi horyzontalnych i wertykalnych.
\FloatBarrier
\begin{figure}[!ht]
    \centering
    \includegraphics[scale=0.6]{Pictures/sobel_filter}
    \caption{Macierze detektora krawędzi Sobela (źródło: opracowanie własne)}
    \label{fig:sobel_filter}
\end{figure}
\FloatBarrier
Zaletą takiego podejścia jest niewątpliwa łatwość użycia, natomiast jedną z głównych wad jest jego wrażliwość na szum.

Często wykorzystywaną metodą do wykrywania krawędzi obiektów w obrazach jest \textit{Binary Image Processing}.
Technika ta polega na sprowadzenia obrazu do postaci, w której kolory pikseli przyjmują tylko dwie wartości - czarną lub białą.
Osiąga się to za pomocą ustalenia progu, który determinuje kolor piksela.
Próg wyznaczany jest na podstawie histogramu obrazu w odcieniach szarości.
Metoda ta jest użyteczna, ze względu na fakt łatwego odseparowania obiektu od tła.
Wykorzystuje ona założenie, że krawędzie tablicy są proste i poziome.
Przy zdeformowanych lub zabrudzonych tablicach, algorytm ten nie osiąga zadowalających wyników.

Inną stosowaną metodą do wykrywania linii na obrazach binarnych jest transformata Hougha \cite{DuanBuildingAA}.
Motywacją do jej opracowania była metoda siłowa (ang. \textit{Brute force}), która jest jednak znacznie bardziej zasobożerna.
Złożoność algorytmu siłowego wynosi $O(n^3)$.
Transformata Hougha polega na twierdzeniu, że każda prosta może być jednoznacznie przedstawiona za pomocą dwóch parametrów.
Przestrzeń tych parametrów to właśnie przestrzeń Hougha.
Najczęściej używanymi parametrami są współczynniki $\rho$ i $\alpha$ z równania prostej w postaci normalnej \eqref{eqn:transform_hough}.
\begin{equation}
    \label{eqn:transform_hough}
    x\cos{\alpha} + y\sin{\alpha} = \rho
\end{equation}
W powyższym równaniu $\rho$ jest promieniem wodzącym, natomiast $\alpha$ kątem tworzonym przez $\rho$ z osią X.
W związku z powyższym, jest to algorytm o liniowej złożoności obliczeniowej.
Można wykazać następujące własności transformacji Hougha:\\
\begin{theorem}
    Prostej przestrzeni kartezjańskiej odpowiada w przestrzeni Hougha punkt, natomiast
    punktowi przestrzeni kartezjańskiej odpowiada w przestrzeni Hougha sinusoidalna krzywa.
    Punkty leżące na tej samej prostej korespondują z sinusoidami przechodzącymi przez
    wspólny punkt w przestrzeni Hougha \cite{hough_transform_definition}.
\end{theorem}
Zasadę transformacji ilustruje rysunek \ref{fig:hough_transform}.
\begin{figure}[!ht]
    \centering
    \includegraphics[scale=1]{Pictures/hough_transform.jpeg}
    \caption{Transformata Hougha (źródło: \cite{Lin:01})}
    \label{fig:hough_transform}
\end{figure}
\FloatBarrier

Innym spotykanym podejściem~\cite{4410602} jest stosowanie dwóch algorytmów.
Pierwszy z nich ma za zadanie wyodrębnić odcinki linii i pogrupować je na podstawie wcześniej ustalonego zbioru warunków geometrycznych.
Drugi znajduje obszary o najwyższym zagęszczeniu pionowych krawędzi.
Dzięki takiemu spojrzeniu na przedstawiony problem, uzyskane wyniki mają wysoką dokładności, szczególnie dla pojazdów znajdujących się w ruchu.
Metode oparte na krawędziach są stosowane w wielu rozwiązaniach ze względu na ich szybkość działania i prostotę.
Jednakże, rozwiązania te są silenie wrażliwe na niepożądane krawędzie i nie sprawdzają się w rozmytych i złożonych obrazach.

\subsection{Metody oparte na kolorach (ang. \textit{Color based)}}
Metody oparte na kolorach bazują na fakcie, że kolor tablicy jest różny od koloru tła pojazdu.
Dla tej grupy rozwiązań, zamiast modelu barw RGB, stosuje się model HSL oparty o nasycenie koloru.
Model ten jest jednak wrażliwy na szum.

Często metody wykorzystujące kolor tablicy rejestracyjnej są używane do wyselekcjonowania kandydatów.
Innymi słowy, oznacza to wybrania obszarów obrazu, w których może znajdować się tablica rejestracyjna.
Technika ta łączona jest z innymi algorytmami, które na kolejnych etapach decydują, czy wskazany obszar rzeczywiście zawiera poszukiwany obiekt.
Do tego typu metod wykorzystywany jest m. in. algorytm \textit{Mean shift} \cite{1520110} i logika rozmyta \cite{Wang2008FuzzybasedAF}.

Opisywana grupa metod może zostać użyta do detekcji zdeformowanych i pochylonych tablic.
Rzadko występują one osobno w metodach detekcji, głównie ze względu na ich dużą czułość na zmiany naświetlenia.
Dodatkowo w zależności od kraju oraz przeznaczenia pojazdu, kolory tablic mogą się znacznie różnić.
Przykładowo obecnie w Polsce tablice aut elektrycznych mają kolor zielony, a samochody zabytkowe żółty (\ref{fig:tablice}).
\FloatBarrier
\begin{figure}[!ht]
    \centering
    \includegraphics[scale=0.6]{Pictures/tablice}
    \caption{Tablice rejestracyjne w Polsce dla aut elektrycznych i zabytkowych (źródło: opracowanie własne)}
    \label{fig:tablice}
\end{figure}
\FloatBarrier

\subsection{Metody oparte na teksturach (ang. \textit{Texture based)}}
Metody oparte na teksturach wykorzystują fakt znajdowania się znaków na tablicach rejestracyjnych.
Znaki na tablicy mają z reguły czarny kolor i znajdują się na jasnym tle tworząc duży kontrast.
Powyższa grupa algorytmów wykorzystuje wysoką częstość zmiany kolorów w obszarze występowania tablic rejestracyjnych.
W \cite{824138} autorzy zaproponowali metodę lokalizacji tablic wykorzystując algorytm kwantyzacji wektorów (ang. \textit{Vector Quantization - VQ}).
W przeciwieństwie do innych metod, które wykorzystywały krawędzi lub kontrast, metoda VQ wykorzystuje aktualną zawartość tablicy rejestracyjnej.
Autorzy wykorzystali bardzo wysoką skuteczność na poziomie 98\%.

W analizie tekstur, często stosuje się filtr Gabora.
Jest to filtr liniowy, pozwalający na przefiltrowanie obrazu z precyzyjnie dobranym zakresem częstotliwości.
Reprezentacje częstotliwości i orientacji filtrów Gabora są uważane przez wielu współczesnych naukowców zajmujących się widzeniem komputerowym za podobne do tych z ludzkiego układu wzrokowego \cite{gabor_human_eye}.
W \cite{gabor_lpr} zaprezentowano algorytm wykorzystujący filtr Gabora.
Jest to jednak metoda czasochłonna i nie znajduje zastosowania dla systemów, w których szybkość działania jest jednym z najistotniejszych czynników.

Wszystkie metody oparte na teksturach są odporne na deformacje tablic.
Jest to kluczowa zaleta ich stosowania.
Mimo to, metody te wymagają skomplikowanych obliczeń i nie dają zadowalających efektów w złożonych środowiskach z różnymi warunkami oświetlenia.

\subsection{Metody oparte na znakach (ang. \textit{Character based)}}
asd


%Ten szablon wykorzystuje pakiet - menedżer spisu literatury o nazwie {\em biber}\footnote{ \url{http://biblatex-biber.sourceforge.net}}, który jest dostępny w wielu popularnych środowiskach programistycznych dla \LaTeX.

%Najpewniejszym sposobem na poprawną kompilację jest wykonanie jej z wiersza poleceń następującą sekwencją rozkazów:
%
%\begin{lstlisting}[language=bash, caption=Skrypt kompilujący, label=alg:1]
%pdflatex main
%biber main
%pdflatex main
%pdflatex main
%
%\end{lstlisting}
%
%Szablon wykorzystuje również rozliczne pakiety, które mogą wymagać
%zaktualizowane do najnowszych wersji. Rekomendowane jest więc, by utrzymywać swoją wersję dystrybucji \LaTeX w najnowszej wersji.
%
%
%\section{Zawartość pakietu}\index{Zawartość pakietu}
%
%Pakiet ze stylem zawiera pliki tekstowe z zawartością przykładowej pracy , czyli w tym wypadku niniejszego przewodnika oraz pliki konfiguracyjne. W tabeli \ref{tab:1} znajduje się lista plików wraz z ich krótkim opisem.
%
%\begin{table}[h]
%    \centering
%    \caption{Pliki w pakiecie szablonu}
%    \begin{tabular}{p{.2\textwidth}|p{.1\textwidth} | p{.60\textwidth}}
%        \toprule
%        \textbf{Nazwa pliku} & \textbf{Edycja} & \textbf{Opis}                                                                                                                                                               \\
%        \midrule
%        main.tex             & TAK             & Główny plik pracy. Zawiera otwarcie dokumentu, dołącza pliki konfiguracyjne, spis treści, dołącza wszystkie rozdziały, literaturę oraz zamyka dokument.                     \\
%        definitions.tex      & TAK             & Zawiera elementy strony tytułowej: autora, tytuł, niezbędne informacje ze strony tytułowej.                                                                                 \\
%        abstract.tex         & TAK             & Zawartość streszczenia pracy w j. polskim i j. angielskim. Zaleca się, by nie przekroczyć tekstem 1 strony.                                                                 \\
%        introduction.tex     & TAK             & Wstęp pracy dyplomowej                                                                                                                                                      \\
%        chapter1.tex         & TAK             & Zawartość rozdziału pierwszego.                                                                                                                                             \\
%        chapter2.tex         & TAK             & Zawartość rozdziału drugiego.                                                                                                                                               \\
%        conclusions.tex      & TAK             & Zakończenie/podsumowanie pracy dyplomowej.                                                                                                                                  \\
%        bibliography.bib     & TAK             & Plik z literaturą w formacie bib.                                                                                                                                           \\
%        Pictures             & TAK             & Katalog zawierający ilustracje i grafiki z pracy. Nazwa katalogu jest wykorzystana w pliku konfiguracyjnym jako katalog domyślny do każdego osadzonego obiektu graficznego. \\
%        title\_page.tex      & NIE             & Struktura strony tytułowej.                                                                                                                                                 \\
%        structure.tex        & NIE             & Plik konfiguracyjny (preambuła). Podłącza i konfiguruje wszystkie niezbędne pakiety oraz definiuje otoczenia wykorzystywane w szablonie.                                    \\
%        \bottomrule
%    \end{tabular}
%    \label{tab:1}
%\end{table}
%------------------------------------------------

%\subsection{Podstawowe wymagania dotyczące szablonu}
%
%Praca ma ustawiony format wielkości strony A4, w pliku {\em structure.tex} w konfiguracji pakietu {\bf geometry}. Przygotowana jest do druku dwustronnego. Ustawiony jest 1cm margines na oprawę.
%
%\noindent{\bf UWAGA!} Każdy rozdział zaczyna się na stronie nieparzystej
%
%\subsection{Bibliografia}
%
%Po zaktualizowaniu pliku z literaturą należy dokonać kompilacji z użyciem procesora {\em biber} a nie standardowego  {\em Bibtex}. Następnie dwukrotnie skompilować z użyciem {\em pdflatex}, aby zmiany były widoczne. Pakiet do zarządzania bibliografią to {\em biblatex}. Jego konfiguracja jest dostępna w pliku {\em structure.tex}
%
%Bibliografia w pracach dyplomowych ma format numeryczny, sortowanie alfabetyczne i jest podzielona na części: książki, artykuły i źródła internetowe i inne. Konfiguracja jest zapisana w niniejszym szablonie.

%Tylko zacytowane prace znajdą się w spisie literatury. Nie należy tego zmieniać. Cytowanie odbywa się przez użycie polecenia $\backslash cite\{\}$ a efekt jest następujący: \cite{book_key}, \cite{article_key}, \cite{lam1994}, \cite{knuthwebsite}.





