%%%%%%%%%%%%%%%%%%%%%%%%%%%%%%%%%%%%%%%%%
% Szablon pracy dyplomowej
% Wydział Informatyki 
% Zachodniopomorski Uniwersytet Technologiczny w Szczecinie
% autor Joanna Kołodziejczyk (jkolodziejczyk@zut.edu.pl)
% Bardzo wczesnym pierwowzorem szablonu był
% The Legrand Orange Book
% Version 2.1 (26/09/2018)
%
% Modifications to LOB assigned by %JK
%%%%%%%%%%%%%%%%%%%%%%%%%%%%%%%%%%%%%%%%%

%----------------------------------------------------------------------------------------
%	CHAPTER 1
% 	author: Joanna Kolodziejczyk (jkolodziejczyk@zut.edu.pl)
%----------------------------------------------------------------------------------------

\chapter{Wstęp}
\label{rozdzial1}

W dzisiejszych czasach ludzka praca stanowi jeden z największych składników kosztów dla wielu przedsiębiorstw.\ Taki stan rzeczy prowadzi do poszukiwania rozwiązań mających na celu zautomatyzowanie najbardziej powtarzalnych czynności.\ Potwierdza to wzrost zainteresowania tematami takimi jak uczenie maszynowe czy widzenie komputerowe na przestrzeni ostatnich lat.\ Jedną z gałęzi gospodarki, w której tego rodzaju automatyzacja jest zauważalna nawet dla osób niezwiązanych z branżą, jest transport drogowy.\ Nieustannie zwiększająca się liczba aut poruszających się po drogach, wzrost sieci dróg i autostrad niejako samoistnie wymusiła próby zautomatyzowania pewnych czynności.


\section{Autorzy stylu}
\index{Autorzy stylu}

Niniejszy szablon powstał na bazie szablonu\footnote{\url{http://www.LaTeXTemplates.com}} o nazwie ,,The Legrand Orange Book''.
Autorzy pierwotnego stylu to:
\begin{itemize}
    \item Mathias Legrand (legrand.mathias@gmail.com)
    \item Vel (vel@latextemplates.com).
\end{itemize}

Późniejsze zmiany na potrzeby stylu Wydziału Informatyki zostały wprowadzone przez Joannę Kołodziejczyk (jkolodziejczyk@zut.edu.pl). Wizualna strona stylu jest konsensusem uzyskanym w wyniku prac zespołu:
\begin{itemize}
    \item Anna Barcz
    \item Marcin Korzeń
    \item Mirosław Łazoryszczak
    \item Piotr Piela
\end{itemize}

Niniejszy styl jest udostępniony na zasadach licencji CC BY-NC-SA 3.0 ({\url{http://creativecommons.org/licenses/by-nc-sa/3.0/}}).


\section{Kompilowanie dokumentu}
\index{Kompilacja}

Do kompilacji wykorzystać należy procesor {\em pdflatex}.

Ten szablon wykorzystuje pakiet - menedżer spisu literatury o nazwie {\em biber}\footnote{ \url{http://biblatex-biber.sourceforge.net}}, który jest dostępny w wielu popularnych środowiskach programistycznych dla \LaTeX.

Najpewniejszym sposobem na poprawną kompilację jest wykonanie jej z wiersza poleceń następującą sekwencją rozkazów:

\begin{lstlisting}[language=bash, caption=Skrypt kompilujący, label=alg:1]
pdflatex main
biber main
pdflatex main
pdflatex main

\end{lstlisting}

Szablon wykorzystuje również rozliczne pakiety, które mogą wymagać
zaktualizowane do najnowszych wersji. Rekomendowane jest więc, by utrzymywać swoją wersję dystrybucji \LaTeX w najnowszej wersji.


\section{Zawartość pakietu}\index{Zawartość pakietu}

Pakiet ze stylem zawiera pliki tekstowe z zawartością przykładowej pracy , czyli w tym wypadku niniejszego przewodnika oraz pliki konfiguracyjne. W tabeli \ref{tab:1} znajduje się lista plików wraz z ich krótkim opisem.

\begin{table}[h]
    \centering
    \caption{Pliki w pakiecie szablonu}
    \begin{tabular}{p{.2\textwidth}|p{.1\textwidth} | p{.60\textwidth}}
        \toprule
        \textbf{Nazwa pliku} & \textbf{Edycja} & \textbf{Opis}                                                                                                                                                               \\
        \midrule
        main.tex             & TAK             & Główny plik pracy. Zawiera otwarcie dokumentu, dołącza pliki konfiguracyjne, spis treści, dołącza wszystkie rozdziały, literaturę oraz zamyka dokument.                     \\
        definitions.tex      & TAK             & Zawiera elementy strony tytułowej: autora, tytuł, niezbędne informacje ze strony tytułowej.                                                                                 \\
        abstract.tex         & TAK             & Zawartość streszczenia pracy w j. polskim i j. angielskim. Zaleca się, by nie przekroczyć tekstem 1 strony.                                                                 \\
        introduction.tex     & TAK             & Wstęp pracy dyplomowej                                                                                                                                                      \\
        chapter1.tex         & TAK             & Zawartość rozdziału pierwszego.                                                                                                                                             \\
        chapter2.tex         & TAK             & Zawartość rozdziału drugiego.                                                                                                                                               \\
        conclusions.tex      & TAK             & Zakończenie/podsumowanie pracy dyplomowej.                                                                                                                                  \\
        bibliography.bib     & TAK             & Plik z literaturą w formacie bib.                                                                                                                                           \\
        Pictures             & TAK             & Katalog zawierający ilustracje i grafiki z pracy. Nazwa katalogu jest wykorzystana w pliku konfiguracyjnym jako katalog domyślny do każdego osadzonego obiektu graficznego. \\
        title\_page.tex      & NIE             & Struktura strony tytułowej.                                                                                                                                                 \\
        structure.tex        & NIE             & Plik konfiguracyjny (preambuła). Podłącza i konfiguruje wszystkie niezbędne pakiety oraz definiuje otoczenia wykorzystywane w szablonie.                                    \\
        \bottomrule
    \end{tabular}
    \label{tab:1}
\end{table}
%------------------------------------------------

\subsection{Podstawowe wymagania dotyczące szablonu}

Praca ma ustawiony format wielkości strony A4, w pliku {\em structure.tex} w konfiguracji pakietu {\bf geometry}. Przygotowana jest do druku dwustronnego. Ustawiony jest 1cm margines na oprawę.

\noindent{\bf UWAGA!} Każdy rozdział zaczyna się na stronie nieparzystej

\subsection{Bibliografia}

Po zaktualizowaniu pliku z literaturą należy dokonać kompilacji z użyciem procesora {\em biber} a nie standardowego  {\em Bibtex}. Następnie dwukrotnie skompilować z użyciem {\em pdflatex}, aby zmiany były widoczne. Pakiet do zarządzania bibliografią to {\em biblatex}. Jego konfiguracja jest dostępna w pliku {\em structure.tex}

Bibliografia w pracach dyplomowych ma format numeryczny, sortowanie alfabetyczne i jest podzielona na części: książki, artykuły i źródła internetowe i inne. Konfiguracja jest zapisana w niniejszym szablonie.

Tylko zacytowane prace znajdą się w spisie literatury. Nie należy tego zmieniać. Cytowanie odbywa się przez użycie polecenia $\backslash cite\{\}$ a efekt jest następujący: \cite{book_key}, \cite{article_key}, \cite{lam1994}, \cite{knuthwebsite}.





