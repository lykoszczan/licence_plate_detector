%----------------------------------------------------------------------------------------
%	CHAPTER 4
%----------------------------------------------------------------------------------------

\chapter{Wyniki badań}
\chaptermark{Wyniki badań} % Tekst, który wyświetli się w nagłówku strony,  jeżeli jest za długi tytuł rozdziałuchapter2.te

\subsection{Wyniki detekcji tablic rejestracyjnych}
TODO - przykład przedstawienia wyników detekcji
\\opis przeprowadzony testów
\\zrzuty z nagrań
\\czasy
\\wyraportowanie błędów
\begin{table}[h]
    \centering
    \caption{Parametry opracowanego zbioru}
    \begin{tabular}{c c c c c c}
        \toprule
        \textbf{\thead{Liczba \\cech}} & \textbf{\thead{Liczba \\słabych \\klasyfikatorów}} & \textbf{\thead{Liczba \\koszyków}} & \textbf{\thead{Dokładność \\klasyfikatora}} & \textbf{\thead{Dokładność \\klasyfikatora dla \\próbek pozytywnych}} & \textbf{\thead{Dokładność \\klasyfikatora dla \\próbek negatywnych}} \\
        \midrule
        2205 & 64 & 8 & 99,8\% & 82,2\% & 99,9\% \\
        \bottomrule
    \end{tabular}
    \label{tab:results_clf}
\end{table}
\\można pokazać zdjęcia nocne
\\duże tablice, małe tablice
\\różny threshold, różne wyniki
\\jako pozytywy wykrywane sa oznaczenia samochodów (modelu)
\\reklamy i banery
\\czas wykonywania
\\krzywe roc

\subsection{Wyniki rozpoznawaniu znaków}
TODO - przedstawienie rozpoznawania na podstawie zdjęć
\\przykłady:
\\mylony znak ``('' z ``J''
\\mylone S z 5