%%%%%%%%%%%%%%%%%%%%%%%%%%%%%%%%%%%%%%%%%
% Plik z wstępem do pracy
% Szablon pracy dyplomowej
% Wydział Informatyki 
% Zachodniopomorski Uniwersytet Technologiczny w Szczecinie
% autor Joanna Kołodziejczyk (jkolodziejczyk@zut.edu.pl)
% Bardzo wczesnym pierwowzorem szablonu był
% The Legrand Orange Book
% Version 2.1 (26/09/2018)
%
% Modifications to LOB assigned by %JK
%%%%%%%%%%%%%%%%%%%%%%%%%%%%%%%%%%%%%%%%%


\chapter*{Wstęp}

%Wstęp powinien być nie dłuższy niż 2 strony. Najlepiej napisać go dopiero, gdy praca jest już skończona i wszystkie jej części spisane.
%
%Wstęp powinien zawierać:
%
%\begin{enumerate}
%\item Opis dziedziny jakiej dotyczy praca, ze wskazaniem, że temat pracy jest ważny, bieżący, itp.
%\item Jaki problem z dziedziny się rozwiązuje.
%\item Cel i teza pracy
%\item W jaki sposób cel zostanie osiągnięty a tez potwierdzona.
%\item Struktura pracy.
%\end{enumerate}

W dzisiejszych czasach ludzka praca stanowi jeden z największych składników kosztów dla wielu przedsiębiorstw.\ Taki stan rzeczy prowadzi do poszukiwania rozwiązań mających na celu zautomatyzowanie najbardziej powtarzalnych czynności.\ Potwierdza to wzrost zainteresowania tematami takimi jak uczenie maszynowe czy widzenie komputerowe na przestrzeni ostatnich lat.\ Jedną z gałęzi gospodarki, w której tego rodzaju automatyzacja jest zauważalna nawet dla osób niezwiązanych z branżą, jest transport drogowy.\ Nieustannie zwiększająca się liczba aut poruszających się po drogach, wzrost sieci dróg i autostrad niejako samoistnie wymusiła próby zautomatyzowania pewnych czynności.\\

Jednym z najczęściej poruszanych zagadnień jest problem rozpoznawania tablic rejestracyjnych (ang. \textit{Licence Plate Recognition} - LPR).\ Do zadań takich systemów należy wykrycie na obrazie obszarów, w których znajdują się tablice rejestracyjne.
Dokładność uzależniona jest od wielu czynników, takich jak jakość obrazu, prędkość pojazdu, warunki atmosferyczne lub pora dnia.
\\\\We wstępie można zawrzeć co jest w dalszych częściach pracy.