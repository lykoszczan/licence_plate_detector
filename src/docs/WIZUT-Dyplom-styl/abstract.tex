%%%%%%%%%%%%%%%%%%%%%%%%%%%%%%%%%%%%%%%%%
% Specjalna strona pracy ze streszczeniem i abstractem w j. angielskim
% Szablon pracy dyplomowej
% Wydział Informatyki 
% Zachodniopomorski Uniwersytet Technologiczny w Szczecinie
% autor Joanna Kołodziejczyk (jkolodziejczyk@zut.edu.pl)
% Bardzo wczesnym pierwowzorem szablonu był
% The Legrand Orange Book
% Version 2.1 (26/09/2018)
%
% Modifications to LOB assigned by %JK
%%%%%%%%%%%%%%%%%%%%%%%%%%%%%%%%%%%%%%%%%


\begin{center}
    \noindent {{\color{blueZUT}\Large\sffamily  {Streszczenie}}}\\[1cm]
\end{center}
Celem niniejszej pracy było opracowanie systemu do rozpoznawania tablic rejestracyjnych pojazdów na obrazach z kamery samochodowej.
Pracę podzielono na część teoretyczną i praktyczną.
W części teoretycznej omówiono wybrane algorytmy z zakresu przetwarzania obrazów i uczenia maszynowego, potrzebne do realizacji postawionego zadania.
W części praktycznej przedstawiono opracowany algorytm oraz zaraportowano wyniki.
Do realizacji zadania detekcji tablic na obrazie, wykorzystano cechy Haara oraz klasyfikator oparty o algorytm RealBoost ze słabymi klasyfikatorami realizowanymi poprzez koszykowanie wartości funkcji logit.
W celu rozpoznania znaków na wykrytych tablicach, użyto biblioteki Tesseract opartej o rekurencyjne sieci neuronowe LSTM\@.
Wyniki badań potwierdzają możliwość zastosowania opracowanego rozwiązania w systemie rozpoznawania tablic rejestracyjnych.
Osiągnięto dokładność równą 99.77\%, przy czym czułość 82\%.
Przetwarzanie jednej klatki obrazu z kamery trwa w przybliżeniu 1 sekundę.

%W tym miejscu trzeba napisać streszczenie pracy w języku polskim.
%Zawiera krótką charakterystykę dziedziny, przedmiotu i wyników zaprezentowanych w pracy. Maksymalnie 1/2 strony.

\vspace{10pt}
\noindent{\bf słowa kluczowe:} uczenie maszynowe, rozpoznawanie tablic rejestracyjnych, obrazy całkowe, widzenie komputerowe

\vfill

\begin{center}
    \noindent {{\color{blueZUT}\Large\sffamily {Abstract}}}\\[1cm]
\end{center}
The aim of this study was to develop a system for recognizing vehicle license plates in images from a car camera.
The thesis was divided into the theoretical and practical parts.
In the theoretical part, selected algorithms in the field of image processing and machine learning, that were needed to complete the task, were described.
The practical part presents the developed algorithm and reports the results.
To complete the task of detecting licence plates in an image, Haar features and a classifier were used based on the RealBoost algorithm with weak classifiers implemented by binning the response of the logit function.
In order to recognize characters on detected plates, the Tesseract library based on recursive LSTM neural networks was used.
The results of tests confirm the possibility of using the developed solution in the vehicle licence plates recognition system.
An accuracy of 99.77\% was achieved, with the sensitivity 82\%.
It takes approximately 1 second to process one frame from the camera image.

\vspace{10pt}
\noindent{\bf keywords:} machine learning, licence plates recognition, integral images, computer vision