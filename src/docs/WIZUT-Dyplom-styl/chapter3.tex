%%%%%%%%%%%%%%%%%%%%%%%%%%%%%%%%%%%%%%%%%
% Szablon pracy dyplomowej
% Wydział Informatyki 
% Zachodniopomorski Uniwersytet Technologiczny w Szczecinie
% autor Joanna Kołodziejczyk (jkolodziejczyk@zut.edu.pl)
% Bardzo wczesnym pierwowzorem szablonu był
% The Legrand Orange Book
% Version 2.1 (26/09/2018)
%
% Modifications to LOB assigned by %JK
%%%%%%%%%%%%%%%%%%%%%%%%%%%%%%%%%%%%%%%%%


%----------------------------------------------------------------------------------------
%	CHAPTER 2
%----------------------------------------------------------------------------------------

\chapter{Opracowany algorytm}
\chaptermark{Opracowany algorytm} % Tekst, który wyświetli się w nagłówku strony,  jeżeli jest za długi tytuł rozdziału


\section{Proces uczenia klasyfikatora}
@todo


\section{Schemat algorytmu}
@todo
diagram graficzny


\section{Biblioteki użyte w programie}
Środowisko Python jest niezwykle popularne m.in. ze względu na mnogość dostępnych gotowych bilbiotek.
Jednym z celów pracy było zaimplementowanie algorytmu uczącego i klasyfikującego.
Cel ten udało się zrealizować, natomiast nie byłoby to możliwe bez użycia gotowych rozwiązań do elementarnych operacji t.j. listowanie plików w katalogach, odczyt i zapis zdjęć, pobieranie pojedynczych klatek z filmu wideo czy operacje na tablicach.
Poniżej wymieniono i opisano najważniejsze oraz najczęściej używane biblioteki w opisywanej pracy.

\subsection{OpenCV}
Open CV jest biblioteką o otwartym źródle (ang. \textit{open source}) \cite{open_cv,open_cv_docs} .
Do jej głównych zastosowań należą przetwarzanie obrazów oraz uczenie maszynowe.
W niniejszej pracy wykorzystano najnowszą dostępną wersję na moment pisania pracy $4.6.0$.
Została zaprojektowana, aby zapewnić ustandaryzowaną infrastrukturę dla aplikacji widzenia komputerowego.
Oprogramowanie dystrybuowane jest na licencji BSD.
Pierwsza wersja została opracowana w roku 1999.
Oryginalnie powstała w języku C++, natomiast istnieją biblioteki pozwalające używać jej w innych językach programowania.
Bibliotekę można podzielić na kilka głównych modułów:
\begin{itemize}
    \item Przetwarzanie obrazów - moduł zawiera zestaw metod do przeprowadzania takich operacji jak filtrowanie, przekształcenia geometryczne, zmiana przestrzeni kolorów, histogramy
    \item Przetwarzanie wideo - zestaw metod, które pozwalają na m. in. usuwanie tła, śledzenie obiektów, wykrywanie ruchu
    \item Operacje wejścia/wyjścia na wideo - wyciąganie poszczególnych klatek z wideo, kodowanie, zapis i odczyt wideo
    \item HighGUI - moduł służący do wizualizacji wyników, wyświetlania okien, zaznaczania ROI (ang. \textit{Region of interest})
\end{itemize}

\subsection{Tesseract OCR}

\subsection{Numpy}

\subsection{Sklearn}

\subsection{Matplotlib}